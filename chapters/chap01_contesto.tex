% !TEX root = ../thesis-main.tex
\chapter{Inquadramento territoriale e scenario di riferimento}
\label{chap:context}

\section{Il Delta del Po e le Valli di Comacchio}
Il Parco del Delta del Po costituisce la più vasta zona umida d’Italia e una delle aree europee a più alta biodiversità, grazie al mosaico di lagune salmastre, valli da pesca, canneti, pinete costiere, dune e zone agricole. Questa varietà di habitat sostiene un numero elevatissimo di specie vegetali e animali, motivo per cui l’area è riconosciuta come Riserva di Biosfera MAB UNESCO \cite{unesco_po_official}.

Dal punto di vista faunistico, il Delta ospita oltre 300--350 specie di uccelli tra nidificanti, svernanti e migratori, rendendolo uno dei siti più importanti d’Europa per l’avifauna acquatica e una meta d'eccellenza per il birdwatching \cite{delta_birdwatching}. Tra le specie più caratteristiche figurano i fenicotteri rosa (\textit{Phoenicopterus roseus}), divenuti simbolo del paesaggio lagunare, insieme ad aironi cenerini, cormorani, cavalieri d’Italia, avocette, falchi di palude e molte altre specie legate alle zone umide \cite{fenicottero_pdf}.

I fenicotteri sfruttano le acque basse e salmastre delle valli e delle lagune per alimentarsi, formando spesso grandi concentrazioni di individui che possono raggiungere diverse migliaia di esemplari nei periodi di maggior presenza \cite{rinnovabili_fenicotteri}. Questa specie è considerata particolarmente protetta a livello internazionale ed è indicativa di buone condizioni ecologiche degli ambienti umidi costieri.

Accanto all’avifauna, il Parco ospita una ricca fauna ittica (circa 60 specie, alcune tipiche del Delta), numerosi invertebrati legati alle acque e ai sedimenti, e una flora dominata da canneti, vegetazione tipica degli ambienti salmastri e resti delle antiche foreste di pianura \cite{delta_species}. Questi elementi contribuiscono a funzioni ecosistemiche chiave come depurazione delle acque, protezione costiera e stoccaggio di carbonio, facendo del Delta del Po un laboratorio naturale per lo studio e la gestione della biodiversità lagunare in un contesto in cui conservazione e attività umane devono coesistere in modo sostenibile \cite{biosfera_mab_reason}.

\subsection{Contesto geologico e storico di formazione}
Il Delta del Po rappresenta un sistema geo-morfologico in continua evoluzione, la cui configurazione attuale deriva dall'interazione tra processi deposizionali fluviali, dinamiche costiere e interventi artificiali come il ``Taglio di Porto Viro'' del 1604, con cui la Serenissima deviò il corso del fiume per ridurre il rischio di insabbiamento della laguna di Venezia, innescando la formazione dell’attuale fronte deltizio \cite{sviluppo_sostenibile}.

\subsection{Caratteristiche climatiche e idrologiche}
Il clima del delta è di tipo temperato continentale con influenza marina attenuata, caratterizzato da precipitazioni medie annue dell’ordine di 700--800 mm, sempre più concentrate in eventi intensi a causa dei mutamenti climatici in atto \cite{turismo_deltadelpo}. Le estati risultano calde e umide, mentre gli inverni sono relativamente miti, con frequenti episodi di nebbia e inversione termica che influenzano la qualità dell’aria e la percezione del paesaggio.

Il sistema idrologico è governato da una fitta rete di canali artificiali di bonifica, gestiti dal Consorzio di Bonifica Delta del Po, che controlla il deflusso verso l'Adriatico e regola i livelli idrici delle valli da pesca e delle aree agricole limitrofe. ARPA Emilia-Romagna coordina una rete integrata di monitoraggio idrometeorologico composta da centinaia di stazioni in telemisura, che acquisiscono in continuo dati su portate, livelli idrometrici e parametri fisico-chimici delle acque superficiali \cite{arpae_monitoraggio_ambientale}.

\subsection{La gestione del territorio e gli strumenti di tutela internazionale}
Il riconoscimento quale Riserva di Biosfera MAB UNESCO sancisce l'eccezionalità di un territorio in cui obiettivi di tutela ambientale e sviluppo socio-economico devono essere integrati in un'unica strategia di lungo periodo \cite{biosfera_mab_reason}. La governance è articolata tra l'Ente di Gestione per i Parchi e la Biodiversità Delta del Po, le Regioni Emilia-Romagna e Veneto e i Comuni ricompresi nell’area del Contratto di Foce, che condividono strumenti pianificatori e misure di gestione coordinate.

In questo quadro istituzionale, la sfida principale consiste nel trovare un equilibrio tra pesca e acquacoltura (in particolare la molluschicoltura), agricoltura intensiva, attività turistiche e necessità di conservazione degli habitat prioritari, in un contesto già reso fragile dai cambiamenti climatici e dalla subsidenza \cite{sviluppo_sostenibile}. Ciò rende particolarmente rilevante l’adozione di strumenti innovativi per monitorare lo stato dell’ecosistema e orientare la fruizione verso forme più consapevoli.

\section{Biodiversità, pressioni antropiche e Smart Tourism}
La biodiversità del Delta del Po, e in particolare delle Valli di Comacchio, rappresenta un patrimonio naturalistico di rilievo internazionale, ma è al tempo stesso esposta a una serie di pressioni antropiche che ne mettono alla prova la resilienza \cite{delta_species}. L’incremento dei flussi turistici, l’intensificazione delle pratiche agricole e le trasformazioni idromorfologiche contribuiscono a modificare gli equilibri ecologici delle lagune e delle zone umide interne.

\subsection{Flora specializzata e habitat di transizione}
La vegetazione del delta è dominata da flora tipica degli ambienti salmastri, adattate a condizioni di stress salino estremo tipiche degli ambienti di transizione tra acqua dolce e marina. Oltre alle salicornie, l'ecosistema lagunare ospita specie come \textit{Limonium serotinum}, \textit{Inula crithmoides} e \textit{Juncus maritimus}, che formano successioni vegetazionali lungo gradienti di idroperiodo e salinità \cite{scienza_della_vegetazione}.

\subsection{Specie simbolo e valore educativo dell’avifauna}
Tra le specie più rappresentative dell’area spicca il fenicottero rosa, che nel corso degli ultimi decenni ha consolidato la propria presenza nelle Valli di Comacchio, diventando un vero e proprio emblema dell’ecosistema lagunare \cite{fenicottero_pdf}. La sua visibilità e il forte impatto iconico rendono questa specie particolarmente efficace come “specie bandiera” per campagne di sensibilizzazione, attività di educazione ambientale e percorsi di visita orientati al birdwatching.

\subsection{Pressioni antropiche e necessità di fruizione consapevole}
Le attività umane che insistono sul Delta del Po comprendono pesca professionale, acquacoltura, agricoltura, turismo balneare e naturalistico, oltre a infrastrutture di trasporto e difesa idraulica. Se da un lato queste attività costituiscono una risorsa economica fondamentale per le comunità locali, dall’altro generano impatti potenziali su habitat e specie, soprattutto in assenza di una pianificazione attenta dei flussi e dei carichi di pressione \cite{turismo_deltadelpo}. In questo scenario emerge la necessità di promuovere forme di fruizione “dolce” del territorio, che coniughino accessibilità, sicurezza e ridotto disturbo per la fauna.

\subsection{Evoluzione della domanda turistica verso la fruizione consapevole}
Negli ultimi anni la domanda turistica nel Delta del Po si è progressivamente orientata verso esperienze più immersive e personalizzate, in cui il visitatore non si limita a “vedere” il paesaggio, ma desidera comprenderne i processi ecologici e le fragilità \cite{smart_tourism}. I visitatori moderni richiedono informazioni puntuali e aggiornate in tempo reale, non solo sui punti di interesse statici, ma anche sulle condizioni ambientali (livelli dell'acqua, stato dei sentieri, meteo locale) per poter pianificare l’escursione in sicurezza e con consapevolezza.

Questo cambiamento si inserisce nel più ampio paradigma dello \textit{Smart Tourism}, in cui le tecnologie digitali svolgono un ruolo chiave nel mettere in relazione dati territoriali, bisogni informativi del turista e obiettivi di sostenibilità della destinazione \cite{smart_tourism}. In questo contesto, strumenti come applicazioni mobili geolocalizzate, piattaforme informative integrate e sistemi di monitoraggio ambientale in tempo reale assumono un’importanza crescente.

\section{Paradigmi per la valorizzazione digitale delle aree protette}
Per rispondere alle sfide descritte è necessario fare riferimento a una serie di paradigmi tecnologici e metodologici che, opportunamente integrati, consentono di progettare nuove forme di esplorazione territoriale. Tra questi assumono particolare rilevanza i sistemi di geolocalizzazione e mappatura dinamica, la \textit{Citizen Science}, le meccaniche di \textit{Gamification} e il concetto di \textit{Digital Twin} ambientale \cite{citizen_science}.

\subsection{Sistemi di geolocalizzazione e mappatura dinamica}
Le mappe digitali hanno progressivamente sostituito la cartografia cartacea nella fruizione turistica e naturalistica delle aree protette, permettendo all’utente di visualizzare in tempo reale la propria posizione e di interagire con punti di interesse (POI) distribuiti sul territorio \cite{wang2012smartphones}. Grazie all’integrazione con i sensori dei dispositivi mobili, è possibile costruire percorsi personalizzati, evidenziare aree di particolare pregio naturalistico e fornire indicazioni contestuali sull’accessibilità dei sentieri o sulle regole di comportamento nelle zone più sensibili \cite{battino2023smart}.

\subsection{Il monitoraggio partecipativo attraverso la Citizen Science}
Il coinvolgimento dei cittadini nella raccolta di dati ambientali, noto come \textit{Citizen Science}, trasforma il visitatore da osservatore passivo a partecipante attivo nel monitoraggio della biodiversità, contribuendo alla ricerca scientifica e alla gestione adattativa del territorio \cite{citizen_science}. Attraverso applicazioni mobili dedicate, i turisti possono segnalare avvistamenti di specie, condizioni dei sentieri o anomalie ambientali, alimentando database condivisi con enti di gestione, ricercatori e comunità locali.

\subsection{La Gamification per l'incentivazione territoriale}
La \textit{Gamification}, intesa come uso di elementi di game design in contesti non ludici, viene sempre più utilizzata per incentivare pratiche sostenibili in ambito educativo, sanitario e ambientale \cite{gamification_def}. In un’ottica di valorizzazione territoriale, l’introduzione di punti esperienza (XP), livelli, badge e missioni consente di rendere più coinvolgente l’esplorazione di luoghi meno conosciuti, distribuendo i flussi turistici e aumentando il tempo di permanenza in aree di interesse naturalistico.

\subsection{Il Digital Twin applicato agli ecosistemi lagunari}
Il paradigma del \textit{Digital Twin} nasce in ambito industriale per descrivere modelli digitali dinamici in grado di riflettere in tempo reale lo stato di un sistema fisico complesso \cite{digital_twin_env}. Applicato agli ecosistemi lagunari, il concetto di gemello digitale permette di collegare dati oggettivi provenienti da sensori idrografici (livello dei canali, salinità, temperatura dell’acqua) con rappresentazioni accessibili all’utente finale, favorendo la comprensione immediata degli equilibri ambientali e delle loro variazioni nel tempo.

\section{EcoSpot: una guida interattiva}
In questo contesto teorico e territoriale si inserisce \textit{EcoSpot}, 
un’applicazione mobile che si propone come guida digitale interattiva per l’esplorazione e la valorizzazione dell’ecosistema del Delta del Po, 
con particolare attenzione alle Valli di Comacchio e alle aree frequentate dal fenicottero rosa. 
L’obiettivo è supportare un modello di turismo lento e consapevole, coniugando scoperta naturalistica, 
monitoraggio ambientale e meccaniche ludiche orientate all’educazione ecologica \cite{turismo_deltadelpo}.

\subsection{Visione e obiettivi dell’applicazione}
EcoSpot aspira al paradigma del ``gemello digitale'' dell’ambiente reale: accompagna l’utente nella scoperta fisica del territorio,
 permettendogli di esplorare habitat, conoscere specie e comprendere lo stato di salute dell’ecosistema attraverso un’interfaccia unificata \cite{digital_twin_env}. L’app è pensata per turisti, famiglie, scuole e appassionati di birdwatching che desiderano un’esperienza immersiva e informata nel Delta del Po.

\subsection{Pilastri progettuali: biodiversità, esplorazione, dati ambientali, gamification}
La progettazione di EcoSpot si fonda su quattro pilastri principali, strettamente interconnessi.

Il primo pilastro è la \textbf{biodiversità come collezione digitale}: l’app mette a disposizione un catalogo strutturato di specie suddivise in categorie 
(uccelli, mammiferi, pesci, aree verdi), in cui ogni scheda fornisce nome comune e scientifico, descrizione sintetica, habitat naturale e curiosità, 
trasformando l’osservazione naturalistica in un’attività di scoperta guidata.

Il secondo pilastro è l’\textbf{esplorazione attiva e il turismo dolce}: grazie alla geolocalizzazione, 
l’utente è incentivato a muoversi fisicamente sul territorio. Per ``scoprire'' una specie o un’area verde e guadagnare punti esperienza è necessario 
avvicinarsi entro un certo raggio a luoghi reali come valli, argini o punti di osservazione, promuovendo un'esplorazione lenta e rispettosa degli habitat.

Il terzo pilastro riguarda il \textbf{“polso” dell’ecosistema in tempo reale}: l’app aggrega dati eterogenei per offrire un quadro completo. 
Da un lato, si interfaccia con la rete di sensori IoT del progetto Discover-ER \cite{discover_er_api} per i parametri idrografici puntuali 
(livello idrometrico, temperatura e conducibilità dell’acqua); dall'altro, integra il servizio Open-Meteo \cite{open_meteo_api} 
per fornire le condizioni e le previsioni meteorologiche locali. In questo modo, fattori normalmente invisibili o dispersi diventano parte integrante 
dell’esperienza di visita unificata.

Il quarto pilastro è la \textbf{gamification educativa}: in accordo con la definizione accademica di utilizzo di elementi di gioco in contesti non ludici 
\cite{gamification_def}, il sistema utilizza XP, livelli e badge per incoraggiare l’utente a esplorare nuovi luoghi e interagire con i dati ambientali.
La progressione da semplice ``Esploratore'' a ``Naturalista'', ``Guardiano'' e infine ``Leggenda del Delta'' è pensata per rafforzare il senso di appartenenza 
 al territorio e la responsabilità verso la sua tutela.

\subsection{Impatto atteso sull’engagement e sulla consapevolezza ecologica}
L’integrazione di questi quattro pilastri mira ad aumentare l’engagement dei visitatori, 
offrendo un’esperienza che combina contenuti informativi, esplorazione fisica e feedback immediati sulle pratiche sostenibili \cite{smart_tourism}. 
Al tempo stesso, la possibilità di accedere a dati ambientali in tempo quasi reale e di collegarli alle osservazioni sul campo contribuisce a sviluppare 
una maggiore consapevolezza ecologica, mostrando in modo concreto la fragilità e il valore degli ecosistemi lagunari. In questo senso, 
EcoSpot si configura non solo come strumento di guida turistica, ma come piattaforma di educazione ambientale e di supporto alla gestione sostenibile delle aree protette del Delta del Po.
