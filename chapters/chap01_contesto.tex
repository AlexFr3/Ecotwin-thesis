% !TEX root = ../thesis-main.tex
\chapter{Contesto del progetto}
\label{chap:context}

\section{Il Parco del Delta del Po e la valorizzazione digitale}
Il Delta del Po rappresenta un ecosistema unico, riconosciuto come Riserva di Biosfera MAB (Man and the Biosphere) dall'UNESCO \cite{unesco_mab}. È caratterizzato da una complessa rete di zone umide che richiedono un costante equilibrio tra conservazione ambientale e attività umane \cite{parcodelta}.

\subsection{Caratteristiche territoriali e ambientali}
L'area è un habitat cruciale per numerose specie, tra cui il Fenicottero Rosa (\textit{Phoenicopterus roseus}), simbolo della biodiversità locale. La flora comprende specie alofile come la Salicornia, adattate agli ambienti salmastri tipici delle lagune costiere.

\subsection{Caratteristiche del lato emiliano-romagnolo}

\paragraph{Il Bosco della Mesola}
Riserva naturale orientata, rappresenta l'ultimo residuo delle antiche foreste termofile litoranee che un tempo ricoprivano la costa adriatica.

\paragraph{Le Valli di Comacchio e le saline}
Un complesso di lagune salmastre dove la gestione idraulica è fondamentale. È qui che il progetto EcoTwin concentra parte del monitoraggio, in particolare nelle stazioni di Bellocchio e Foce, aree strategiche per l'avifauna \cite{parcodelta}.

\paragraph{I lidi e la costa emiliano-romagnola}
Aree soggette a forte pressione antropica, che necessitano di strumenti innovativi per veicolare i flussi turistici verso forme di fruizione più sostenibili e consapevoli.

\subsection{Esigenze informative e turistiche}
I visitatori moderni richiedono informazioni puntuali e in tempo reale. Nell'ottica dello \textit{Smart Tourism} \cite{smart_tourism}, non è sufficiente indicare i punti di interesse statici; è necessario fornire dati contestuali sulle condizioni ambientali (es. livelli dell'acqua, meteo specifico) per pianificare l'escursione in sicurezza.

\section{Utilizzo della Tecnologia nel Settore Turistico-Naturalistico}

\subsection{Applicazioni mobile e mappe interattive}
Le mappe digitali hanno progressivamente sostituito quelle cartacee, permettendo la geolocalizzazione dinamica dell'utente e l'interazione con i Punti di Interesse (POI) distribuiti sul territorio.

\subsection{Citizen Science}
Il coinvolgimento dei cittadini nella raccolta dati, noto come \textit{Citizen Science} \cite{citizen_science}, trasforma il turista da osservatore passivo a partecipante attivo nel monitoraggio della biodiversità, contribuendo alla ricerca scientifica.

\subsection{Gamification}
La \textit{Gamification} è definita come l'uso di elementi di game design in contesti non ludici \cite{gamification_def}. Nel progetto EcoTwin, meccaniche quali punti esperienza (XP), livelli e badge (es. "Leggenda del Delta") vengono impiegate per incentivare l'esplorazione fisica di nuovi habitat.

\section{Il progetto EcoTwin (DISCOV.ER)}

\subsection{Obiettivi del progetto}
L'obiettivo principale è la creazione di un "Digital Twin" \cite{digital_twin_env} informativo che colleghi i dati oggettivi dei sensori idrografici (livello canali, salinità) all'esperienza soggettiva del visitatore.

\subsection{Implementazione tecnologica}
L'architettura proposta si basa su un'applicazione mobile sviluppata in Flutter \cite{flutter}, collegata a un backend Cloud per la sincronizzazione dei progressi e a un'infrastruttura IoT per l'acquisizione dei dati ambientali in tempo reale.

\subsection{Risultati attesi}
Si attende un aumento dell'engagement dei visitatori e una maggiore consapevolezza ecologica, dimostrando come le tecnologie digitali possano supportare efficacemente la gestione e la fruizione delle aree protette.