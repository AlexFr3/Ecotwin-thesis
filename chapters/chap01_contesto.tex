% !TEX root = ../thesis-main.tex
\chapter{Contesto e Background}
\label{chap:context}

\section{Il Delta del Po e le Valli di Comacchio}
Il Parco del Delta del Po costituisce la più vasta zona umida d’Italia e una delle aree europee a più alta biodiversità, grazie al mosaico di lagune salmastre, valli da pesca, canneti, pinete costiere, dune e zone agricole. Questa varietà di habitat sostiene un numero elevatissimo di specie vegetali e animali, motivo per cui l’area è riconosciuta come Riserva di Biosfera MAB UNESCO \cite{unesco_po_official}.

Dal punto di vista faunistico, il Delta ospita oltre 300--350 specie di uccelli tra nidificanti, svernanti e migratori, rendendolo uno dei siti più importanti d’Europa per l’avifauna acquatica e una meta d'eccellenza per il birdwatching \cite{delta_birdwatching}. Tra le specie più caratteristiche figurano i fenicotteri rosa (\textit{Phoenicopterus roseus}), divenuti simbolo del paesaggio lagunare, insieme ad aironi cenerini, cormorani, cavalieri d’Italia, avocette, falchi di palude e molte altre specie legate alle zone umide \cite{fenicottero_pdf}.
\begin{figure}[H]
    \centering
    \begin{subfigure}[b]{0.45\textwidth}
        \centering
        \includegraphics[width=\textwidth]{figures/animals/blackwinged_stilt.jpg}
        \caption{Cavaliere d'Italia (\textit{Himantopus himantopus})}
        \label{fig:cavaliere}
    \end{subfigure}
    \hfill
    \begin{subfigure}[b]{0.3\textwidth}
        \centering
        \includegraphics[width=\textwidth]{figures/animals/cormorant.jpg}
        \caption{Cormorano (\textit{Phalacrocorax carbo})}
        \label{fig:cormorano}
    \end{subfigure}
    
    \vspace{0.5cm}
    
    \begin{subfigure}[b]{0.3\textwidth}
        \centering
        \includegraphics[width=\textwidth]{figures/animals/avocet.jpg} 
        \caption{Avocetta (\textit{Recurvirostra avosetta})}
        \label{fig:avocetta}
    \end{subfigure}
    \hfill
    \begin{subfigure}[b]{0.45\textwidth}
        \centering
        \includegraphics[width=\textwidth]{figures/animals/marsh_harrier.jpg}
        \caption{Falco di palude (\textit{Circus aeruginosus})}
        \label{fig:falco}
    \end{subfigure}
    
    \caption[Avifauna del Delta]{Esemplari rappresentativi dell'avifauna del Delta del Po. (Fonte: Pixabay, img. Royalty-Free)}
    \label{fig:altri_uccelli}
\end{figure}
I fenicotteri sfruttano le acque basse e salmastre delle valli e delle lagune per alimentarsi, formando spesso grandi concentrazioni di individui che possono raggiungere diverse migliaia di esemplari nei periodi di maggior presenza \cite{rinnovabili_fenicotteri}. Questa specie è considerata particolarmente protetta a livello internazionale ed è indicativa di buone condizioni ecologiche degli ambienti umidi costieri.

Accanto all’avifauna, il Parco ospita una ricca fauna ittica (circa 60 specie, alcune tipiche del Delta), numerosi invertebrati legati alle acque e ai sedimenti, e una flora dominata da canneti, vegetazione tipica degli ambienti salmastri e resti delle antiche foreste di pianura \cite{delta_species}. Questi elementi contribuiscono a funzioni ecosistemiche chiave come depurazione delle acque, protezione costiera e stoccaggio di carbonio, facendo del Delta del Po un laboratorio naturale per lo studio e la gestione della biodiversità lagunare in un contesto in cui conservazione e attività umane devono coesistere in modo sostenibile \cite{biosfera_mab_reason}.

\subsection{Storia, idrogeologia e clima}
Il Delta del Po rappresenta un sistema geo-morfologico in continua evoluzione, la cui configurazione attuale deriva dall'interazione tra processi deposizionali fluviali, dinamiche costiere e interventi artificiali come il ``Taglio di Porto Viro'' del 1604, con cui la Serenissima deviò il corso del fiume per ridurre il rischio di insabbiamento della laguna di Venezia, innescando la formazione dell’attuale fronte deltizio \cite{sviluppo_sostenibile}.

Il clima del delta è di tipo temperato continentale con influenza marina attenuata, caratterizzato da precipitazioni medie annue dell’ordine di 700--800 mm, sempre più concentrate in eventi intensi a causa dei mutamenti climatici in atto \cite{turismo_deltadelpo}. Le estati risultano calde e umide, mentre gli inverni sono relativamente miti, con frequenti episodi di nebbia e inversione termica che influenzano la qualità dell’aria e la percezione del paesaggio.

Il sistema idrologico è governato da una fitta rete di canali artificiali di bonifica, gestiti dal Consorzio di Bonifica Delta del Po, che controlla il deflusso verso l'Adriatico e regola i livelli idrici delle valli da pesca e delle aree agricole limitrofe. ARPA Emilia-Romagna coordina una rete integrata di monitoraggio idrometeorologico composta da centinaia di stazioni in telemisura, che acquisiscono in continuo dati su portate, livelli idrometrici e parametri fisico-chimici delle acque superficiali \cite{arpae_monitoraggio_ambientale}.

\subsection{Vulnerabilità idraulica: il problema del cuneo salino}
Un fattore critico per la sopravvivenza degli habitat del Delta è l'intrusione salina, ovvero la risalita delle acque marine lungo l'alveo fluviale durante i periodi di magra. Negli ultimi anni, a causa della siccità prolungata e dell'abbassamento delle portate del fiume Po, il cuneo salino è risalito per oltre 30-40 km dalla foce \cite{cuneo_salino_report}.

Questo fenomeno altera drasticamente la chimica delle acque, minacciando l'agricoltura e modificando la distribuzione delle specie ittiche e dell'avifauna. La necessità di monitorare parametri come la conducibilità elettrica (Indicatore della salinità) non è quindi solo un'esigenza scientifica, ma diventa un'informazione di interesse pubblico per comprendere le mutazioni del paesaggio in tempo reale. L'integrazione di questi dati in applicazioni consumer permette di rendere visibile l'invisibile, sensibilizzando i visitatori su dinamiche complesse spesso ignorate dal turismo di massa.
\subsection{Gestione e tutela del territorio}
Il riconoscimento quale Riserva di Biosfera MAB UNESCO sancisce l'eccezionalità di un territorio in cui obiettivi di tutela ambientale e sviluppo socio-economico devono essere integrati in un'unica strategia di lungo periodo \cite{biosfera_mab_reason}. La governance è articolata tra l'Ente di Gestione per i Parchi e la Biodiversità Delta del Po, le Regioni Emilia-Romagna e Veneto e i Comuni ricompresi nell’area del Contratto di Foce, che condividono strumenti pianificatori e misure di gestione coordinate.
\begin{figure}[H]
    \centering
    \includegraphics[width=0.6\textwidth]{figures/green_area/MappaParcoDeltaDelPo.pdf} 
    \caption[Parco del Delta del Po]{(Fonte: Riserva di Biosfera Delta del Po MAB UNESCO (biosferadeltapo.it))}
    \label{fig:Mappa parco del Delta del Po}
\end{figure}

In questo quadro istituzionale, la sfida principale consiste nel trovare un equilibrio tra pesca e acquacoltura (in particolare la molluschicoltura), agricoltura intensiva, attività turistiche e necessità di conservazione degli habitat prioritari, in un contesto già reso fragile dai cambiamenti climatici e dalla subsidenza \cite{sviluppo_sostenibile}. Ciò rende particolarmente rilevante l’adozione di strumenti innovativi per monitorare lo stato dell’ecosistema e orientare la fruizione verso forme più consapevoli.
%controllare
\subsection{Il versante emiliano-romagnolo}
Il versante emiliano-romagnolo del Delta del Po riunisce, in pochi chilometri, ambienti molto diversi tra loro: boschi planiziali relitti, vaste valli salmastre, saline costiere e un sistema di lidi sabbiosi che si affacciano sull’Adriatico \cite{parcodelta}. 
Questa eterogeneità paesaggistica configura l’area come scenario privilegiato per iniziative di valorizzazione ambientale ad alto valore aggiunto.

\subsubsection{C\`a Vecchia e Pineta San Vitale}
Oltre alle valli e ai boschi relitti, il versante emiliano-romagnolo del Delta del Po comprende numerosi punti di osservazione attrezzati per il birdwatching,
 spesso posizionati lungo argini o all’ingresso di aree umide di particolare pregio \cite{birdwatching_cavecchia}. 
 Tra questi si segnalano la zona di C\`a Vecchia, nei pressi della Pineta di San Vitale e della Pialassa Baiona, 
 dove capanni e percorsi schermati consentono di osservare in modo discreto colonie di ardeidi, anatidi e limicoli. 
 Luoghi di questo tipo rappresentano contesti ideali per integrare l’esperienza sul campo con strumenti digitali, 
 che possono guidare l’utente nell’individuazione dei punti migliori di osservazione e nella comprensione delle specie incontrate.
\begin{figure}[H]
    \centering
    \includegraphics[width=0.6\textwidth]{figures/green_area/PinetaSanVitale.jpg} 
    \caption[Pineta di San Vitale]{(Fonte: Regione Emilia-Romagna, ambiente.regione.emilia-romagna.it, licenza CC-BY 4.0)}
    \label{fig:Pineta San Vitale}
\end{figure}

\subsubsection{Pineta di Classe}
A sud delle Valli di Comacchio, in direzione Ravenna e Cervia, la Pineta di Classe rappresenta uno dei più estesi polmoni verdi della Riviera Adriatica, con circa 900 ettari di superficie boscata compresi tra la località di Classe e la foce del torrente Bevano \cite{pineta_classe}. Inserita nel Parco regionale del Delta del Po e designata come sito di interesse comunitario e zona di protezione speciale, la pineta conserva un mosaico di habitat che alternano formazioni a pino domestico e leccio, radure sabbiose, acquitrini e aree salmastre connesse alle vicine paludi dell’Ortazzo e dell’Ortazzino.

Dal punto di vista della fruizione, la Pineta di Classe è attraversata da una fitta rete di sentieri ciclopedonali e 
percorsi che consentono di esplorare l’ambiente con punti di osservazione affacciati sulle zone umide frequentate da una ricca presenza di volatili \cite{pineta_classe}. 
Questa combinazione di bosco costiero e aree allagate ne fa un contesto ideale per itinerari naturalistici e attività di birdwatching.
\begin{figure}[htbp]
    \centering
    \includegraphics[width=0.8\textwidth]{figures/green_area/Pineta_di_Classe.jpg}
    \caption[Pineta di Classe]{Scorcio della Pineta di Classe. 
    (Fonte: Wikimedia Commons, foto di Jimi thing, licenza CC BY-SA 3.0)}
    \label{fig:pineta_classe}
\end{figure}
\subsubsection{Bosco della Mesola}
Il Gran Bosco della Mesola rappresenta uno degli ultimi e meglio conservati boschi planiziali litoranei dell’Italia settentrionale,
testimonianza delle antiche foreste che un tempo ricoprivano la costa adriatica \cite{bosco_mesola}. Con oltre 1.000 ettari di estensione, 
è caratterizzato dall'alternanza di habitat forestali e radure umide che costituiscono rifugio per una fauna di pregio, tra cui il celebre cervo della Mesola, 
ultimo nucleo autoctono di cervo padano, oltre a numerose specie di uccelli, rettili e anfibi. L’accesso regolamentato e la presenza di percorsi dedicati 
rendono il bosco un luogo ideale per attività guidate di educazione ambientale e per itinerari in bicicletta o a piedi.
\begin{figure}[htbp]
    \centering
    \includegraphics[width=0.6\textwidth]{figures/green_area/Bosco_Mesola.jpg}
    \caption[Bosco della Mesola]{Scorcio del Gran Bosco della Mesola.
    (Fonte: Wikimedia Commons, foto di Marco Walker, licenza CC BY-SA 3.0)}
    \label{fig:bosco_mesola}
\end{figure}
\subsubsection{Valli e Saline di Comacchio}
Le Valli di Comacchio costituiscono il più vasto complesso di zone umide salmastre dell’Emilia-Romagna,
 con oltre 14.000 ettari di specchi d’acqua poco profondi, arginature e canali che definiscono un paesaggio lagunare unico \cite{valli_comacchio_ambiente}. 
 Il sistema vallivo è suddiviso in sottobacini interconnessi, tra cui Valle Fattibello, Valle Campo e la Fossa di Porto, caratterizzati da fondali argillosi e
  salinità variabile, che creano condizioni favorevoli per comunità alofile specializzate e per una ricchissima avifauna acquatica.

All’interno di questo contesto si inseriscono le Saline di Comacchio, un’area umida di elevato valore naturalistico dove la gestione dei livelli idrici e 
della salinità ha favorito, storicamente, sia la produzione del sale sia la presenza di grandi contingenti di uccelli acquatici, tra cui il fenicottero rosa 
e numerose specie di limicoli \cite{valli_comacchio_ambiente}. Anche le saline rappresentano oggi uno dei principali hotspot per il birdwatching dell’area, 
grazie alla presenza di argini sopraelevati e punti di osservazione schermati.
\begin{figure}[htbp]
    \centering
    \includegraphics[width=0.7\textwidth]{figures/green_area/Saline_di_Comacchio.jpg}
    \caption[Saline di Comacchio]{La Torre Rossa nelle Saline di Comacchio.
    (Fonte: Wikimedia Commons, foto di Luukas, Pubblico Dominio)}
    \label{fig:saline_torrerossa}
\end{figure}

\subsubsection{Lidi costieri}
I Lidi ferraresi si sviluppano per circa 25 chilometri lungo la costa adriatica e fanno parte integrante del territorio del Parco regionale del Delta del Po, combinando spiagge sabbiose a bassa pendenza, pinete retrodunali e sistemi di dune più o meno consolidate \cite{lidi_ferraresi_turismo}. Lungo questo tratto di litorale, località come Lido di Volano, Lido delle Nazioni e Lido degli Estensi costituiscono punti di partenza privilegiati per itinerari cicloturistici e pedonali che collegano il mare alle valli interne, favorendo un modello di turismo balneare integrato con la scoperta degli ambienti umidi retrostanti.
\begin{figure}[htbp]
    \centering
    \includegraphics[width=0.8\textwidth]{figures/green_area/Lido_di_Spina.jpg}
    \caption[Spiaggia naturale dei Lidi]{Tratto di spiaggia naturale presso la Riserva del Lido di Spina.
    (Fonte: Wikimedia Commons, foto di Wilfred Krause, licenza CC BY-SA 3.0)}
    \label{fig:lido_spina_natura}
\end{figure}


\section{Biodiversità, pressioni antropiche e  turismo}
La biodiversità del Delta del Po, e in particolare delle Valli di Comacchio, rappresenta un patrimonio naturalistico di rilievo internazionale, ma è al tempo stesso esposta a una serie di pressioni antropiche che ne mettono alla prova la resilienza \cite{delta_species}. L’incremento dei flussi turistici, l’intensificazione delle pratiche agricole e le trasformazioni idromorfologiche contribuiscono a modificare gli equilibri ecologici delle lagune e delle zone umide interne.
\begin{figure}[H]
    \centering
    \begin{subfigure}[b]{0.48\textwidth}
        \centering
        \includegraphics[width=\textwidth]{figures/green_area/vallidicomacchio.jpg}
        \label{fig:vallidicomacchio}
    \end{subfigure}
    \hfill
    \begin{subfigure}[b]{0.48\textwidth}
        \centering
        \includegraphics[width=\textwidth]{figures/green_area/vallidicomacchio2.jpg}
        \label{fig:vallidicomacchio2}
    \end{subfigure}
    \caption[Valli di Comacchio]{(Fonte: Gettyimages, img. Royalty-Free)}
    \label{fig:Comacchio}
\end{figure}
\subsection{Flora e fauna caratteristica}
La vegetazione del delta è dominata da flora tipica degli ambienti salmastri, adattate a condizioni di stress salino estremo tipiche degli ambienti di transizione tra acqua dolce e marina. Oltre alle salicornie, l'ecosistema lagunare ospita specie come \textit{Limonium serotinum}, \textit{Inula crithmoides} e \textit{Juncus maritimus}, che formano successioni vegetazionali lungo gradienti di idroperiodo e salinità \cite{scienza_della_vegetazione}.

Tra le specie più rappresentative dell’area spicca il fenicottero rosa, che nel corso degli ultimi decenni ha consolidato la propria presenza nelle 
Valli di Comacchio, diventando un vero e proprio emblema dell’ecosistema lagunare \cite{fenicottero_pdf}. La sua visibilità e il forte impatto iconico 
rendono questa specie particolarmente efficace come “specie bandiera” per campagne di sensibilizzazione, attività di educazione ambientale e percorsi di 
visita orientati al birdwatching.
\begin{figure}[H]
    \centering
    \includegraphics[width=0.6\textwidth]{figures/animals/pink_flamingo.jpg} 
    \caption[Il Fenicottero Rosa]{Esemplare di fenicottero. Questa specie (\textit{Phoenicopterus roseus}) ha colonizzato stabilmente le Valli di Comacchio dagli anni '90. (Fonte: Pixabay, img. Royalty-Free)}
    \label{fig:fenicottero}
\end{figure}
\clearpage\subsection{Evoluzione dell'offerta turistica}

Le attività antropiche che insistono sul Delta del Po dalla pesca professionale all'acquacoltura, fino all'agricoltura e ai 
sistemi di bonifica e protezione costituiscono una risorsa economica imprescindibile per le comunità locali. 
Tuttavia, tali attività generano inevitabili pressioni su habitat e specie, rendendo urgente una pianificazione attenta dei 
carichi antropici per evitare il degrado degli ecosistemi \cite{turismo_deltadelpo}.

In risposta a queste criticità, si è resa necessaria una transizione dell'offerta turistica verso modelli di \textit{ecoturismo} e 
turismo lento (\textit{slow tourism}). Questi approcci non si limitano a promuovere una fruizione ``dolce'' e a basso impatto ambientale, 
ma mirano a valorizzare le specificità locali, incoraggiando tempi di visita dilatati che favoriscano una connessione profonda con il territorio, 
contrapponendosi alla logica del turismo di massa ``mordi e fuggi''.

Parallelamente a questa necessità di tutela, si osserva una profonda evoluzione nella domanda. Negli ultimi anni, l'interesse dei visitatori nel 
Delta del Po si è progressivamente orientato verso esperienze più immersive e personalizzate: il turista contemporaneo non si limita a voler ``vedere'' 
il paesaggio, ma desidera comprenderne i processi ecologici, le tradizioni e le fragilità \cite{smart_tourism}.
Questa nuova consapevolezza trasforma il visitatore da spettatore passivo a soggetto attivo, che richiede informazioni puntuali e 
aggiornate in tempo reale come i livelli idrometrici, la percorribilità dei sentieri o le condizioni meteo locali per pianificare le proprie 
escursioni in sicurezza e autonomia.

È in questo punto di incontro, tra l'esigenza di tutela del patrimonio e la ricerca di esperienze consapevoli, 
che si inserisce il paradigma dello \textit{Smart Tourism}. Le tecnologie digitali assumono qui un ruolo abilitante fondamentale: 
strumenti come applicazioni mobili geolocalizzate e sistemi di monitoraggio \ac{IoT} non servono solo a migliorare l'esperienza utente, 
ma diventano veicoli strategici per orientare i flussi turistici, educare alla sostenibilità e monitorare lo stato di salute dell'ambiente in tempo reale \cite{smart_tourism}.
\section{Valorizzazione digitale delle aree protette}
Per rispondere alle sfide descritte nelle sezioni precedenti è necessario fare riferimento a una serie di paradigmi tecnologici e metodologici che, opportunamente integrati, consentono di progettare nuove forme di esplorazione territoriale. Tra questi assumono particolare rilevanza i sistemi di geolocalizzazione e mappatura dinamica, la \textit{Citizen Science}, le meccaniche di \textit{Gamification} e il concetto di \textit{Digital Twin} ambientale \cite{citizen_science}.

\subsection{Geolocalizzazione}
Le mappe digitali hanno progressivamente sostituito la cartografia cartacea nella fruizione turistica e naturalistica delle aree protette, permettendo all’utente di visualizzare in tempo reale la propria posizione e di interagire con punti di interesse (POI) distribuiti sul territorio \cite{wang2012smartphones}. Grazie all’integrazione con i sensori dei dispositivi mobili, è possibile costruire percorsi personalizzati, evidenziare aree di particolare pregio naturalistico e fornire indicazioni contestuali sull’accessibilità dei sentieri o sulle regole di comportamento nelle zone più sensibili \cite{battino2023smart}.

\subsection{Citizen Science}

Il termine \textit{Citizen Science} indica, in senso ampio, 
il coinvolgimento attivo di persone non professioniste in una o più fasi del processo di ricerca scientifica: 
dalla formulazione delle domande di ricerca, alla raccolta, analisi e interpretazione dei dati, fino alla 
diffusione dei risultati e, in alcuni casi, alla definizione di politiche e azioni di gestione \cite{jcom_citizen_science}. 
In altre parole, si tratta di attività scientifiche condotte ``con'' o ``da'' il pubblico, spesso in collaborazione 
con istituzioni di ricerca, enti di gestione e comunità locali, con l'obiettivo di produrre nuova conoscenza basata 
su evidenze scientifiche e di rafforzare il legame tra scienza e società.

Una delle definizioni più diffuse, ripresa anche dall'\textit{Oxford English Dictionary} e declinata in italiano, 
descrive la citizen science come ``attività scientifica condotta da membri del pubblico, spesso in collaborazione con 
o sotto la direzione di scienziati professionisti e istituzioni scientifiche''. 
La Commissione Europea la definisce come il coinvolgimento volontario di scienziati non professionisti in diverse 
fasi delle attività di ricerca e innovazione, con livelli di partecipazione variabili dalla definizione delle agende 
di ricerca e di policy, alla raccolta e analisi dei dati, fino alla valutazione dei risultati \cite{erc_citizen_science}.

In Italia, reti e iniziative dedicate (ad esempio \textit{Citizen Science Italia}) sottolineano in particolare la natura 
collaborativa di tali processi, intesi come ricerca svolta con il contributo attivo di cittadini e scienziati per generare 
nuova conoscenza utile alla società e ai decisori pubblici \cite{csi_italia}.

Le pratiche di citizen science possono assumere forme molto diverse: progetti \textit{contributivi} in cui i volontari 
si occupano soprattutto della raccolta di dati secondo protocolli standardizzati; progetti \textit{collaborativi} in cui 
i cittadini partecipano anche alla definizione dei metodi e all'interpretazione dei risultati; progetti \textit{co-creati}, 
ideati e sviluppati congiuntamente da comunità locali e ricercatori, spesso con una forte valenza di democrazia partecipativa 
e giustizia ambientale \cite{citizen_science}.

In ambito ambientale e di conservazione, la citizen science rappresenta uno strumento strategico per il monitoraggio su 
larga scala di specie e habitat, per la valutazione dello stato degli ecosistemi e per l'orientamento di politiche di 
gestione adattativa del territorio.
Nel contesto del Delta del Po, le applicazioni digitali dedicate e le piattaforme di \textit{crowdsourcing} dei dati consentono 
di coinvolgere visitatori, residenti e portatori di interesse in attività che vanno ben oltre la semplice segnalazione di 
avvistamenti faunistici o delle condizioni dei sentieri. I cittadini possono contribuire alla documentazione di specie rare 
o invasive, al monitoraggio di fenomeni idrologici ed erosivi, alla registrazione di pressioni antropiche e impatti climatici, 
fornendo informazioni preziose per enti gestori, ricercatori e amministrazioni locali. In questo senso, la citizen science 
non solo trasforma il visitatore da osservatore passivo a partecipante attivo, ma diventa un dispositivo di coprogettazione 
della conoscenza e di corresponsabilità nella tutela e nella gestione sostenibile del territorio deltizio.
\subsection{Gamification}
La \textit{Gamification}, intesa come uso di elementi di game design in contesti non ludici, viene sempre più utilizzata per incentivare pratiche 
sostenibili in ambito educativo, sanitario e ambientale \cite{gamification_def}. In un’ottica di valorizzazione territoriale, l’introduzione di punti 
esperienza (XP), livelli, badge e missioni consente di rendere più coinvolgente l’esplorazione di luoghi meno conosciuti, distribuendo i flussi turistici 
e aumentando il tempo di permanenza in aree di interesse naturalistico. L'efficacia della \textit{Gamification} in ambito ambientale non risiede però 
solo nell'assegnazione di punti o badge, ma trova fondamento nella Teoria dell'Autodeterminazione (SDT - Self-Determination Theory) \cite{ryan_deci_sdt},
secondo cui la motivazione intrinseca di un utente viene stimolata soddisfacendo tre bisogni psicologici fondamentali: 
 \begin{itemize}
    \item \textbf{Competenza}, ossia la sensazione di acquisire nuove abilità (es. imparare a riconoscere un fenicottero o un'avocetta tramite l'app);
    \item \textbf{Autonomia}, ovvero la libertà di esplorare il territorio scegliendo i propri percorsi e obiettivi, senza costrizioni rigide; 
    \item \textbf{Relazione}, cioè il senso di appartenenza a una comunità (es. la community degli utenti di EcoSpot che contribuiscono alla tutela del parco).
 \end{itemize} 
Nel contesto del turismo sostenibile, le meccaniche di gioco devono quindi essere progettate non per banalizzare l'esperienza naturale, 
ma per fornire feedback immediati che gratifichino comportamenti virtuosi, come l'esplorazione di aree meno affollate o la segnalazione corretta di specie
faunistiche.

\subsection{Digital Twin}
Una definizione ampiamente citata in letteratura \cite{glaessgen2012digital}, 
proposta da Glaessgen e Stargel in ambito aerospaziale, descrive il Digital Twin come una 
«simulazione multifisica, multiscala, probabilistica e ad altissima fedeltà che riflette, 
in maniera tempestiva, lo stato del corrispondente gemello fisico, basandosi su dati storici, 
dati in tempo reale provenienti da sensori e su un modello fisico».
Questa definizione sottolinea tre elementi chiave: l'esistenza di un'entità fisica, di una 
controparte virtuale e di un flusso di dati continuo che le collega, permettendo di osservare, 
comprendere e prevedere il comportamento del sistema reale in condizioni operative complesse.

Applicato agli ecosistemi lagunari, il paradigma del gemello digitale consente di collegare 
dati oggettivi provenienti da sensori idrografici (livello dei canali, salinità, temperatura dell'acqua) 
con rappresentazioni accessibili all'utente finale. Questo approccio non si limita a mostrare i dati, 
ma favorisce una comprensione immediata degli equilibri ambientali e delle loro variazioni nel tempo, 
rendendo ``visibile'' ciò che spesso rimane nascosto sotto la superficie dell'acqua.
\section{Analisi dello stato dell'arte e soluzioni esistenti}
\label{sec:state_of_art}
Per comprendere appieno il posizionamento di una nuova soluzione digitale, è fondamentale analizzare il panorama delle applicazioni esistenti dedicate al turismo naturalistico e al monitoraggio ambientale. Attualmente, le soluzioni disponibili possono essere classificate in tre macro-categorie.

\subsection{Piattaforme di Citizen Science}
Applicazioni come \textit{iNaturalist} rappresentano lo standard globale per la raccolta di dati sulla biodiversità \cite{inaturalist_ref}. Queste piattaforme permettono agli utenti di caricare foto e identificare specie, generando enormi database utili alla ricerca scientifica. Tuttavia, il loro approccio è spesso ``data-centric'': l'interfaccia è complessa per l'utente occasionale, manca una componente ludica strutturata e non vi è integrazione con i dati ambientali non vivent (come i livelli idrici o il meteo locale), elementi cruciali per la fruizione del Delta del Po.

\subsection{Guide turistiche generaliste}
Applicazioni come \textit{Google Maps}, \textit{TripAdvisor} o le app turistiche regionali offrono eccellenti funzionalità di navigazione e recensione dei servizi 
(ristorazione, alloggio). Tuttavia, queste soluzioni mancano di specificità ambientale: un sentiero in zona A (tutela integrale) viene spesso trattato come una qualsiasi attrazione turistica, senza fornire all'utente il contesto comportamentale necessario o le informazioni sulla fauna visibile in quella specifica stagione.

\subsection{Applicazioni verticali di parco}
Esistono diverse applicazioni sviluppate specificamente per singoli parchi nazionali o regionali. 
L'analisi di questi software evidenzia spesso due criticità: la scarsa manutenzione nel tempo 
(molte app non vengono aggiornate dopo la fine del finanziamento del progetto) e la mancanza di interattività in tempo reale. 
Spesso si configurano come brochure digitali statiche, prive di integrazione con sensori \ac{IoT}. o meccaniche di coinvolgimento attivo dell'utente.
\textit{EcoSpot} si inserisce in questo vuoto funzionale, 
proponendo un modello ibrido che unisce l'immediatezza di consultazione delle guide turistiche, 
il rigore della Citizen Science e l'accessibilità in tempo reale ai dati di monitoraggio.

\subsection{Il progetto DISCOV.ER}
Un esempio applicativo di rilievo, che integra le tecnologie descritte in questo capitolo, è rappresentato dal progetto \textit{DISCOV.ER} (Digital Twin and Citizen Science for the Sustainability of Areas of Natural and Tourist Interest).
Presentato alla conferenza internazionale \textit{GoodIT '25}, il progetto propone un approccio innovativo per la gestione e la valorizzazione delle aree di interesse naturale e turistico \cite{2025discover}.

L'obiettivo principale di DISCOV.ER è sfruttare la sinergia tra il paradigma del \textit{Digital Twin} e la metodologia della \textit{Citizen Science} per promuovere la sostenibilità ambientale. Il sistema si basa sull'idea che un gemello digitale non debba essere solo una replica statica dell'ambiente, ma uno strumento dinamico alimentato sia da dati provenienti da sensori IoT, sia dalle osservazioni raccolte direttamente dai cittadini e dai turisti.
Questa integrazione permette di:
\begin{itemize}
    \item \textbf{Monitorare lo stato dell'ambiente:} Utilizzando dati eterogenei per rilevare in tempo reale cambiamenti o criticità nell'ecosistema.
    \item \textbf{Coinvolgere la comunità:} Trasformando i visitatori in partecipanti attivi (\textit{human-as-a-sensor}), aumentando la loro consapevolezza ecologica e il senso di appartenenza al territorio.
    \item \textbf{Supportare una gestione adattativa:} Permettendo agli enti di rispondere rapidamente ai cambiamenti rilevati nel sistema socio-ecologico attraverso strumenti analitici avanzati.
\end{itemize}

Il progetto DISCOV.ER dimostra come l'integrazione fluida (\textit{seamless integration}) tra indagine scientifica e coinvolgimento sociale possa generare valore sia per la ricerca che per la tutela del patrimonio naturale, ponendosi come riferimento diretto per lo sviluppo di soluzioni come EcoSpot.
\section{Obiettivi del progetto di tesi}
\textit{EcoSpot} si inserisce in questo vuoto funzionale, proponendo un modello ibrido che unisce l’immediatezza di consultazione delle guide turistiche, 
il rigore della \textit{Citizen Science} e l’accessibilità in tempo reale ai dati di monitoraggio. 
In questo contesto teorico e territoriale, l'applicazione si configura come una guida digitale interattiva per l’esplorazione e la valorizzazione 
dell’ecosistema del Delta del Po, con particolare attenzione alle Valli di Comacchio e alle aree frequentate dal fenicottero rosa.

L’obiettivo principale è supportare un modello di Ecoturismo avanzato, coniugando scoperta naturalistica, monitoraggio ambientale e meccaniche ludiche 
orientate all’educazione ecologica \cite{turismo_deltadelpo}.
EcoSpot aspira ad accompagnare l’utente nella scoperta fisica del territorio, permettendogli di esplorare habitat, conoscere specie e comprendere lo 
stato di salute dell’ecosistema attraverso un’interfaccia unificata. L’app è pensata per turisti, famiglie, scuole e appassionati di \textit{birdwatching} 
che desiderano un’esperienza immersiva e informata nel Delta del Po.
\subsection{Obiettivi progettuali}
La progettazione di EcoSpot è guidata da quattro obiettivi, volti a coniugare l'esperienza turistica con la consapevolezza ecologica.

\begin{itemize}
    \item \textbf{Divulgazione del patrimonio naturalistico:} l'obiettivo è diffondere la conoscenza della biodiversità locale attraverso una \textit{collezione digitale} strutturata. L'applicazione offre un catalogo dettagliato di specie (suddivise in categorie come uccelli, mammiferi, pesci e aree verdi), corredato da nomi scientifici, descrizioni e habitat, trasformando la semplice osservazione in un percorso di scoperta guidata e consapevole.

    \item \textbf{Incentivazione del turismo lento:} il sistema mira a favorire un'esplorazione rispettosa del territorio sfruttando meccaniche \textit{location-based}. Grazie alla geolocalizzazione, l'utente è incentivato a raggiungere fisicamente luoghi reali (valli, argini, punti di osservazione) per sbloccare i contenuti, promuovendo così una mobilità dolce e un presidio attivo degli habitat, in contrapposizione al turismo di massa.

    \item \textbf{Monitoraggio ambientale integrato:} il progetto intende rendere accessibili parametri ecologici complessi tramite l'aggregazione di dati 
    in tempo reale. L'app si interfaccia con le \ac{API} del progetto DISCOV.ER \cite{discover_er_api} per i dati idrografici (livello, temperatura, 
    conducibilità) e integra il servizio Open-Meteo \cite{open_meteo_api} per le condizioni atmosferiche, rendendo visibili all'utente fattori altrimenti 
    impercettibili dell'ecosistema.

    \item \textbf{Incremento del coinvolgimento:} l'ultimo obiettivo è massimizzare la partecipazione dell'utente attraverso la \textit{gamification}. In linea con la definizione accademica \cite{gamification_def}e l'uso di elementi ludici come punti esperienza (XP), livelli (da ``Esploratore'' a ``Leggenda del Delta'') serve a rafforzare il senso di appartenenza al territorio e a stimolare comportamenti virtuosi di tutela ambientale.
\end{itemize}
L’integrazione di questi quattro pilastri mira ad aumentare il coinvolgimento dei visitatori, 
offrendo un’esperienza che combina contenuti informativi, esplorazione fisica e feedback immediati sulle pratiche sostenibili \cite{smart_tourism}. 
Al tempo stesso, la possibilità di accedere a dati ambientali in tempo quasi reale e di collegarli alle osservazioni sul campo contribuisce a sviluppare 
una maggiore consapevolezza ecologica, mostrando in modo concreto la fragilità e il valore degli ecosistemi lagunari. In questo senso, 
EcoSpot si configura non solo come strumento di guida turistica, ma come piattaforma di educazione ambientale e di supporto alla gestione sostenibile delle aree protette del Delta del Po.
