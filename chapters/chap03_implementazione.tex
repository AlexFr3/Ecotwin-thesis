%!TEX root = ../thesis-main.tex
\chapter{Analisi e Implementazione}
\label{chap:implementation}

Questo capitolo descrive le componenti core dell'applicazione: la gestione della mappa, il sistema di gamification e l'integrazione dei sensori.

\section{Mappa Interattiva e Clustering}
La visualizzazione cartografica è gestita dal widget \texttt{MapWidget}, che integra la libreria \texttt{flutter\_map} \cite{fluttermap}. Per gestire l'elevato numero di marker (animali, piante, sensori) e mantenere fluide le prestazioni, è stato implementato un sistema di clustering tramite il pacchetto \texttt{flutter\_map\_marker\_cluster}.

Il seguente codice mostra come vengono renderizzati i marker per le aree verdi utilizzando i widget nativi di Flutter \cite{flutter}:

\begin{lstlisting}[language=Dart, caption={Rendering dei marker sulla mappa}, label={lst:map_markers}]
// Esempio estratto da MapWidget
...widget.greenAreas.map((area) {
  return Marker(
    point: area.centerPoint,
    width: 70,
    height: 70,
    child: GestureDetector(
      onTap: () {
        showModalBottomSheet(
          context: context,
          builder: (context) => GreenAreaDetailSheet(species: area),
        );
      },
      child: Image.asset(area.imagePath),
    ),
  );
})
\end{lstlisting}

\section{Sistema di Gamification}
Il cuore dell'esperienza utente è il sistema di scoperta. L'app calcola costantemente la distanza geodetica tra la posizione dell'utente e quella delle specie target.

\subsection{Logica di Scoperta}
Quando la distanza rilevata è inferiore al parametro \texttt{discoveryRadius}, viene invocato il metodo \texttt{discoverSpecies} nel database helper. L'algoritmo è implementato in Dart \cite{dart} sfruttando la precisione del GPS.

\begin{lstlisting}[language=Dart, caption={Algoritmo di scoperta basato sulla distanza}, label={lst:discovery}]
void _checkDiscoveries(LatLng userLocation) {
  // Calcolo distanza
  final double meters = _distanceCalculator.as(
      LengthUnit.Meter, userLocation, species.centerPoint);

  if (meters <= species.discoveryRadius) {
    _triggerDiscovery(user.uid, species);
  }
}
\end{lstlisting}

\subsection{Gestione Utente e Rango}
Il modello utente (\texttt{UserModel}) calcola dinamicamente il livello e il rango (es. "Guardiano", "Leggenda") in base agli XP accumulati, aggiornando lo stato dell'applicazione in tempo reale.

\section{Integrazione Dati Sensori (IoT)}
Il servizio \texttt{SensorService} interroga l'API AWS \cite{aws} ed effettua il parsing dei dati JSON ricevuti dalle stazioni di monitoraggio (es. Bellocchio, Foce).

\begin{lstlisting}[language=Dart, caption={Parsing dei dati sensore in Dart}, label={lst:sensor_parsing}]
final response = await http.post(Uri.parse(_apiUrl), ...);

if (response.statusCode == 200) {
  final decodedBody = jsonDecode(response.body);
  // Estrazione valore sensore
  value = decodedBody['sensorValue'] ?? decodedBody['value'];
  return MapEntry(sensorName, value ?? "N/D");
}
\end{lstlisting}

Questi dati vengono poi presentati all'utente cliccando sulle icone delle stazioni sulla mappa, fornendo un quadro in tempo reale dello stato di salute del Parco.